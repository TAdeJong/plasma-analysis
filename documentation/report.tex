\documentclass{article}
\usepackage{amsmath, amssymb, textcomp}
\usepackage{graphicx}
\usepackage{cite}
\usepackage[hidelinks]{hyperref}
\usepackage[section]{placeins}

\newcommand{\Part}[3][ ]{\ensuremath{\frac{\partial^{#1} #2}{{\partial #3}^{#1}}}}
\newcommand{\Dif}[3][ ]{\ensuremath{\frac{d^{#1} #2}{{d #3}^{#1}}}}
\renewcommand{\O}[1]{\ensuremath{\mathcal{O}\left(#1\right)}}
\renewcommand{\vec}{\bold}

\begin{document}
\title{Plasma Analysis}
\author{Tobias de Jong \& David Kok}
\date{\today}
\maketitle
\section{Introduction}
{\bf Fysische achtergrond, motivatie voor numeriek. Bonuspunten voor gebruik van het woord Topologisch }
We will use CUDA to efficiently calculate the topological properties of magnetic field lines as generated by Chris Smiet's simulations.\cite{PhysRevLett.115.095001}\\
To implement the code we were inspired by Marek Fiser.\footnote{\url{http://www.marekfiser.com/Projects/Real-time-visualization-of-3D-vector-field-with-CUDA}}
\section{Theory}
{\bf plasma-fluidity, field lines }
\section{Numerical theory}
{\bf RK4, datastructuren (+textures), algoritme ontwerp + complexiteit in termen van relevante parameters, optimalisaties?}
\section{Results}
{\bf plaatjes, timings + scaling, portability, meer plaatjes: lines, lengths, windings, masking}
\section{Discussion}
{\bf Algoritmekeuzes \textrightarrow betere keuzes, sign flip in center, stitching issues, aannames over geometrie}
\section{Suggestions for future work}
{\bf betere algoritmes: vanwege memory bound meer rekenen tijdens RK4, normaal implementeren, betere modulariteit, dynamische stapgrootte. Variantie/FFT op windingsgetallen in tijdsdomein}
\bibliographystyle{plain}
\bibliography{introduction}
\end{document}
