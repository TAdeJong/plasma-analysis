\documentclass{article}
\usepackage{amsmath, amssymb, textcomp}
\usepackage{graphicx}
\usepackage{cite}
\usepackage{hyperref}
\usepackage[section]{placeins}

\newcommand{\Part}[3][ ]{\ensuremath{\frac{\partial^{#1} #2}{{\partial #3}^{#1}}}}
\newcommand{\Dif}[3][ ]{\ensuremath{\frac{d^{#1} #2}{{d #3}^{#1}}}}
\renewcommand{\O}[1]{\ensuremath{\mathcal{O}\left(#1\right)}}
\renewcommand{\vec}{\bold}

\begin{document}
\title{Plasma Analysis}
\author{Tobias de Jong \& David Kok}
\date{\today}
\maketitle
\section{Introduction}
We will use CUDA to efficiently calculate the topological properties of magnetic field lines as generated by Chris Smiet's simulations.
\section{Data representation}
The dataset we use is a set of $256^3$ points in a regular rectangular grid for which the $(x,y,z)$ components of the magnetic field are given. This sort of dataset has a strong correspondence with a 3D texture on a GPU. Using this representation in CUDA has several advantages, of which the hardware trilinear interpolation is the most notable.
\section{Coordinate systems}
Troughout the analysis we use three coordinate systems: cartesian $9x,y,z)$, spherical $(r,\theta,\phi)$ and toroidal $(s,\alpha, \beta)$. We use cartesian as the original dataset is cartesian and this yields the simplest line integrators. For the analysis it will however be beneficial to think in toroidal coordinates:
\[(s,\alpha,\beta)\in ([0,R_max],[0,2\pi],[0,2\pi])\]
where $\alpha$ is the angle in the horizontal plane, corresponding to $\theta$ in spherical coordinates, $s$ is the (shortest) distance to the core of the torus, and $\beta$ is the remaining angle on the torus, in the $r,\phi$-plane.


\end{document}
